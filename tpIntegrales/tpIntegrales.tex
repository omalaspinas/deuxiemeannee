\documentclass[a4paper,10pt]{article}
\usepackage[utf8]{inputenc}
\usepackage[french]{babel}
\usepackage{amsfonts,bm,amsmath,amssymb,graphicx}
\usepackage{cancel}

\newcommand{\dd}{\mathrm{d}}
\newcommand{\real}{\mathbb{R}}
\newcommand{\integer}{\mathbb{Z}}
\newcommand{\definition}{\textbf{Definition }}
\newcommand{\exemples}{\textbf{Exemples }}
\newcommand{\remarque}{\textbf{Remarque }}
\newcommand{\proprietes}{\textbf{Propriétés }}
\newcommand{\propriete}{\textbf{Propriété }}

\title{Travail pratique sur les intégrales}
% \author{Orestis Malaspinas}
\date{A rendre pour le XX.YY.2016}

\begin{document}
\maketitle

Le but de ce travail pratique est d'implanter les méthodes numériques de calcul d'intégrales que nous avons vues en cours,
afin de les comprendre de façon un peu plus approfondie.

Dans un premier temps, le but est donc d'écrire un code où l'utilisateur spécifie une fonction $f(x)$ qu'on 
suppose ``gentille'' (pas besoin de vérifier 
si elle est bien définie partout par exemple), un intervalle $[a,b]$, et 
un nombre de subdivisions $N$. Le code devra rendre la valeur numérique obtenue pour l'intégrale de la fonction $I(a,b,N,f(x))$ 
pour trois méthodes 
vues en cours (méthode du rectangle à gauche, méthode du trapèze et méthode de Simpson).

Puis vous devrez effectuer une étude de l'erreur pour chacune de ces méthodes. Il s'agira de prendre une fonction $f(x)$ 
dont la primitive est simple à calculer, disons
\begin{equation}
f(x)=\frac{1}{x},
\end{equation}
et un intervalle sur lequel la fonction est bien définie. Choisissons ici $[a,b]$ avec $a=1$ et $b=5$. 
On peut donc calculer l'intégrale exactement et on notera ce résultat exact $I_{exact}(a,b,f(x))$.
Il s'agira de calculer l'erreur commise par l'évaluation de la fonction $I(a,b,N,f(x))$ de la façon suivante
\begin{equation}
 E(N)=\left|\frac{I_{exact}(a,b,f(x))-I(a,b,N,f(x))}{I_{exact}(a,b,f(x))}\right|
\end{equation}
Ces résultats devront être illustrés sous forme de graphique ($E$ en fonction de $N$ en échelle log-log).

Finalement, une comparaison des performances des différentes méthodes devra être effectuée.
On choisira des $\varepsilon=0.1,0.01,0.001$ (au sens du cours) pour savoir si la convergence de la méthode est atteinte.
On comparera le temps qu'il faut pour calculer l'intégrale avec les différentes méthodes
avec une résolution permettant d'avoir atteint la convergence pour chaque $\varepsilon$
(à présenter sous forme de tableau). 
Rappelons ici que nous avons convergence si pour un $N$ donné, on a
\begin{equation}
 \left|\frac{I(a,b,N,f(x))-I(a,b,2\cdot N,f(x))}{I(a,b,2\cdot N,f(x))}\right|<\varepsilon.
\end{equation}

Vous devrez rendre un petit rapport (3-4 pages) qui explique ce que vous avez fait et dans quel but. Il devra contenir
une courte introduction théorique (rappelant les formules et le but du travail), une partie expliquant dans les grandes lignes 
l'algorithme (pas de copier-coller du code), une partie illustrant les résultats obtenus, et finalement
une conclusion résumant les résultats.

Le travail peut-être effectué en groupe de deux, mais les rapports doivent être individuels 
(le code peut être identique, n'oubliez pas de mentionner 
explicitement si vous avez effectué le code à deux).  Je dois pouvoir exécuter le code
afin de pouvoir reproduire les résultats présentés dans le rapport. Je dois aussi pouvoir 
définir ma propre fonction à intégrer de façon simple.
Vous pouvez m'envoyer le rapport au format pdf et le code par e-mail.

La note sera une combinaison entre le code rendu et le rapport (moitié/moitié). 

\end{document}